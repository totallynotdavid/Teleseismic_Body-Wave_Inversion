\documentclass[spanish,a4paper,11pt]{article}
\usepackage[utf8]{inputenc}
\usepackage[T1]{fontenc}
\usepackage[spanish]{babel}
\usepackage{booktabs}
\usepackage{dcolumn}
\usepackage{bm}
\usepackage{fancyhdr}
\RequirePackage{graphicx}
\RequirePackage[absolute]{textpos}

\usepackage[colorlinks,linkcolor={blue},citecolor={blue},urlcolor={red}]{hyperref}
\hypersetup{urlcolor=blue, colorlinks=true} % Colors hyperlinks in blue

\linespread{1.1} % Interlineado
\usepackage{geometry} % Control de margenes
\geometry{
	left=25mm,
	right=25mm,
	top=40mm,
	bottom=25mm
}

\begin{document}
\thispagestyle{empty}

\addtolength{\topmargin}{-1cm}
\addtolength{\headsep}{-1cm}
\enlargethispage{0.5cm}
\addtolength{\footskip}{0.5cm}

%\logo{UNMSM}
\begin{textblock*}{297mm}(11mm,16mm)
	\includegraphics[scale=0.28]{unmsm-logo}
\end{textblock*}

\par
\centerline{\large \textbf{UNIVERSIDAD NACIONAL MAYOR DE SAN MARCOS}}
\vspace{8pt}
\centerline{\large\textbf{FACULTAD DE CIENCIAS FÍSICAS}}
\vspace{8pt}
\centerline{\large\textbf{Escuela Profesional de Física}}
\vspace{8pt}
\centerline{\large\textbf{PROYECTO DE TESIS DE LICENCIATURA }}
\vspace{10pt}


\section{Responsabilidad de Ejecución} \noindent 
Tesista: Juan Perez \\
Asesor: Juan Rodriguez  \\


\section{Título del Trabajo:} \noindent 
Las Ecuaciones de Lagrange  

\section{Resumen } \noindent 
En este trabajo se calculan las ecuaciones ...

\section{Contenido} 

\subsection{Introducción} 
\subsection*{Antecedentes de trabajos previos}
...

\subsection*{Planteamiento del problema}
...

\subsection*{Planteamiento de la hipótesis}
...

\subsection*{Objetivos} 
...

\subsection{Fundamentación Teórica} 
...

\subsection{Metodología} 
...

\subsection{Cronograma} 
...

\section{Nivel del Trabajo de Investigación:} \noindent 
Trabajo de Tesis para obtener el Título Profesional en Física.  \\


\begin{thebibliography}{99}
\bibitem[Ape18]{Ape2018} Apellido, N. (2018). Ejemplo de Artículo en Formato APA. \emph{Rev. Inv. Fis., 21}(1), pp 18-26.
	
\bibitem[Ben18]{Ben2018} Benny, H. y Pérez, J. (2018). \emph{Título de Libro en Formato APA}. Lima: Editorial San Marcos.
	
\bibitem[Cro92]{Cro1992} Cros, A. y Muret, P. (1992). Properties of noble-metal/silicon junctions, \emph{Mater. Sci. Rep.} \textbf{8}, 271. 
	
\bibitem[Ide07]{Ide2007} Ide, S. (2007). Slip Inversion. \emph{Treatise on Geophysics}. En H. Kanamori (Ed.), \emph{Treatise on Geophysics} (pp 193-223). Amsterdam: Elsevier Science.

\bibitem[Kik03]{Kik2003} Kikuchi, M. and Kanamori, H. (2003) Notes on Teleseismic Body-Wave Inversion Program, web:
\url{http://www.eri.u-tokyo.ac.jp/ETAL/KIKUCHI/}
	
\bibitem[Oka92]{Oka1992} Okada, Y. (1992). Internal deformation due to shear and tensile faults in a half space. \emph{BSSA}, Vol. 82, No. 2, pp 1018-1040.
	
\bibitem[Pul99]{Pul1999} Pulker, H. (1999). \emph{Coatings On Glass}, Segunda edición. Amsterdam: Elsevier Science.
	
\end{thebibliography}

\centerline{\rule{100mm}{0.2mm}}

\vspace{40pt}
\hspace{280pt} Lima, 20 de julio de 2021

\vspace{100pt}
\centerline{............................................. \hspace{100pt}  .............................................}

\centerline{(Tesista) \hspace{200pt} (Asesor)}


\end{document}
